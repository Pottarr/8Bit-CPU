\documentclass[12pt]{report}

\usepackage[a4paper, margin=2.5cm]{geometry}
\usepackage{array}
\usepackage{amsmath, amssymb}
\usepackage{fancyhdr}
\usepackage{graphicx}
\usepackage{hyperref}
\usepackage{indentfirst}
\usepackage{listings}
\usepackage{lipsum}
\usepackage{minted}
\usepackage{rotating}
\usepackage{tabularx}
\usepackage{titlesec}
\usepackage{tocloft}

\setlength{\parindent}{1cm}
\setlength{\parskip}{5pt}

\titlespacing*{\chapter}{0pt}{0.5cm}{0.5cm}
\titlespacing*{\section}{0pt}{0.25cm}{0.25cm}

\setlength{\cftbeforetoctitleskip}{1cm}
\setlength{\cftaftertoctitleskip}{1cm}

\begin{document}

\begin{titlepage}
	\centering
    \vspace*{\fill}
	\includegraphics[width=0.3\textwidth]{./images/KMITL Logo.png} \\
	\vspace{1cm}
	{\LARGE \textbf{8 Bits CPU Simulation Documentation}} \\[0.5cm]
	\vspace{0.5cm}
	{\large \textbf{Compter Architecture and Organization}} \\[0.5cm]
	{\large \textbf{Software Engineering Program,}} \\[0.5cm]
	{\large \textbf{Department of Computer Engineering,}} \\[0.5cm]
	{\large \textbf{School of Engineering, KMITL}} \\[1cm]
	{\Large 67011093 Chavit Sarutdeechaikul} \\[0.25cm]
    {\Large 67011352 Theepakorn Phayonrat}
    \vspace*{5cm}
    \vspace*{\fill}
\end{titlepage}

% Preface page
\chapter*{Preface}

This project, Design and Implementation of a Minimal 8-Bit CPU, was
undertaken as part of the Computer Architecture and Organization course
in the second year of Bachelor of Software Engineering at KMITL.
It represents a practical exploration of fundamental computer architecture
concepts, from logic design to instruction execution. The project allowed
us to apply theoretical knowledge of CPU structure, instruction sets,
and pipelining into a fully simulated and functioning processor. Using
the Digital simulator by Hneemann, we developed and verified a modular
CPU that embodies the essence of real processor operation. This report
documents the design process, implementation details, and lessons
learned throughout the development cycle.

\newpage

% Abstract page
\chapter*{Abstract}

This mini project focuses on the design and simulation of a minimal
8-bit pipelined CPU using the Digital circuit simulation software.
The CPU incorporates a five-stage pipeline: Fetch, Decode, Execute,
Memory, and Write Back—and supports a custom instruction set architecture
(ISA) with arithmetic, logical, branching, and I/O operations.
The architecture features separate ROM and RAM modules, basic input
and output ports, and a status register for flag management. Through
simulation and test programs, the system demonstrates instruction
execution, branching control, and interrupt handling. The project
enhances understanding of CPU internals, digital logic integration,
and the challenges of pipelined design, offering a hands-on approach 
to fundamental architectural concepts.

\newpage

% Table of contents
\tableofcontents
\newpage

% Chapter 1: Introduction
\chapter{Introduction}
\section{Project Overview}

The 8-Bit CPU Simulation project aims to implement a simplified,
educational model of a pipelined processor that captures the essential
operations of modern CPUs while remaining manageable for simulation.
The design follows a five-stage pipeline (Instruction Fetch,
Instruction Decode, Execute, Memory, Write Back), allowing concurrent 
instruction processing for improved efficiency.

The CPU operates with 8-bit data paths and addresses, includes
general-purpose and a special register (flag status register), and
separates program memory (ROM) from data memory (RAM). A custom
instruction set defines all operations, including data transfer
(LD, ST, MOV), arithmetic and logic (ADD, SUB, MUL, AND, OR, XOR, NOT),
branching (BZ, BNZ, BC, B), and I/O handling (RD, WR). Interrupt support
and optional features such as basic caching or branch prediction may 
extend functionality.

Simulation and verification are performed in Digital, emphasizing
modularity, waveform analysis, and clear documentation.

\newpage

\section{Background}

Central Processing Units (CPUs) are the computational core of digital
systems, responsible for executing instructions and managing data flow
between memory and peripherals. Understanding how a CPU functions—from
fetching an instruction to writing back results—is crucial in computer
engineering education.

Traditional classroom learning often focuses on theoretical aspects
such as microarchitecture, ISA design, and pipelining, but lacks
direct visualization of hardware behavior. This project bridges that
gap by using the Digital simulator to implement a CPU from basic logic
components. By designing each pipeline stage, students explore how
control signals, registers, buses, and memory interact to form a
functioning processor.

The project serves as a foundational experience in CPU design, 
illustrating key topics such as instruction decoding, data hazards,
memory access timing, and modular circuit organization.


\section{Objective}

\begin{enumerate}
    \item \textbf{To design and implement} an 8-bit pipelined CPU with a custom instruction set architecture (ISA) using the \textit{Digital} simulation environment.
    \item \textbf{To apply theoretical concepts} of CPU architecture, including pipeline stages, ALU operations, branching, and flag management, in a practical design.
    \item \textbf{To develop modular circuit components} (e.g., ALU, control unit, registers, and memory interfaces) that integrate seamlessly within the pipeline.
    \item \textbf{To simulate and verify} the CPU’s operation through test programs demonstrating arithmetic, logic, branching, stack, and interrupt handling.
    \item \textbf{To enhance understanding} of digital logic design, data path organization, and hardware–software interaction in CPU execution.
    \item \textbf{To document} the architecture, instruction formats, and verification results clearly and comprehensively for academic evaluation.
\end{enumerate}

\chapter{Project Overview}
\section{Hardware Design}

\subsection{Structure}


\includegraphics[width=0.6\textwidth]{images/app_overview.png}

\newpage

\subsection{Pipeline States}

\subsection{Pipeline Hazards Management}

\newpage

\section{Software Design}

\subsection{Instructions}

\noindent Instruction Format (Length: 20 bits): \\
$$Opcode [19:15]\ |\ IMM_{Flag} [14]\ |\ R_{Dest}\ [13:11]\ |\ R_{Src1}\ |\ R_{Src2} [2:0]\ or\ IMM [7:0]$$

\begin{center}
\scalebox{0.6}{
\[
\setlength{\extrarowheight}{5pt} % Adds 5pt to each row's height
\begin{array}{|l|l|l|}
\hline
\textbf{Instruction} & \textbf{Opcode + Fields} & \textbf{Description} \\[4pt]
\hline
\text{Do in } IF & & \\
NOP & 00000\ |\ IMM_{Flag} [14]\ |\ DC[13:0] & \text{No Operation} \\[4pt]
\hline
\text{Do in } EX_{ALU} & & \\
ADD & 00100\ |\ IMM_{Flag} [14]\ |\ R_{Dest} [13:11]\ |\ R_{Src1} [10:8]\ |\ R_{Src2} [2:0]\ or\ \#IMM [7:0] & \text{Addition} \\[4pt]
SUB & 00101\ |\ IMM_{Flag} [14]\ |\ R_{Dest} [13:11]\ |\ R_{Src1} [10:8]\ |\ R_{Src2} [2:0]\ or\ \#IMM [7:0] & \text{Subtraction} \\[4pt]
MUL & 00110\ |\ IMM_{Flag} [14]\ |\ R_{Dest} [13:11]\ |\ R_{Src1} [10:8]\ |\ R_{Src2} [2:0]\ or\ \#IMM [7:0] & \text{Multiplication} \\[4pt]
AND & 01000\ |\ IMM_{Flag} [14]\ |\ R_{Dest} [13:11]\ |\ R_{Src1} [10:8]\ |\ R_{Src2} [2:0]\ or\ \#IMM [7:0] & \text{Bitwise AND} \\[4pt]
OR  & 01001\ |\ IMM_{Flag} [14]\ |\ R_{Dest} [13:11]\ |\ R_{Src1} [10:8]\ |\ R_{Src2} [2:0]\ or\ \#IMM [7:0] & \text{Bitwise OR} \\[4pt]
XOR & 01010\ |\ IMM_{Flag} [14]\ |\ R_{Dest} [13:11]\ |\ R_{Src1} [10:8]\ |\ R_{Src2} [2:0]\ or\ \#IMM [7:0] & \text{Bitwise XOR} \\[4pt]
NOT & 01011\ |\ IMM_{Flag} [14]\ |\ R_{Dest} [13:11]\ |\ DC [10:8]\ |\ R_{Src2} [2:0]\ or\ \#IMM [7:0] & \text{Bitwise NOT} \\[4pt]
\hline
\text{Do in } EX_{JUMP} & & \\
BC & 10000\ |\ IMM_{Flag} [14]\ |\ DC [13:8]\ |\ R_{Src2} [2:0]\ or\ \#IMM [7:0] & \text{Branch if Carry} \\[4pt]
BZ  & 10001\ |\ IMM_{Flag} [14]\ |\ DC [13:8]\ |\ R_{Src2} [2:0]\ or\ \#IMM [7:0] & \text{Branch if Zero} \\[4pt]
BNZ & 10010\ |\ IMM_{Flag} [14]\ |\ DC [13:8]\ |\ R_{Src2} [2:0]\ or\ \#IMM [7:0] & \text{Branch if Not Zero} \\[4pt]
BNG & 10011\ |\ IMM_{Flag} [14]\ |\ DC [13:8]\ |\ R_{Src2} [2:0]\ or\ \#IMM [7:0] & \text{Branch if Negative} \\[4pt]
B & 10100\ |\ IMM_{Flag} [14]\ |\ DC [13:8]\ |\ R_{Src2} [2:0]\ or\ \#IMM [7:0] & \text{Unconditional Branch} \\[4pt]
\hline
\text{Do in } ME_{Stack} & & \\
PSH & 10110\ |\ DC [14:3]\ |\ R_{Src2} [2:0] & \text{Push } R_{Src2} \text{ value into Stack} \\[4pt]
POP & 10111\ |\ DC [14:3]\ |\ R_{Src2} [2:0] & \text{Pop the top register value from the Stack} \\[4pt]
\hline
\text{Do in } ME_{I/O} & & \\
RD  & 11000\ |\ DC [14:3]\ |\ R_{Src2} [2:0] & \text{Read input value to register} \\[4pt]
WR  & 11001\ |\ DC [14:3]\ |\ R_{Src2} [2:0] & \text{Write value from register to output} \\[4pt]
\hline
\text{Do in } ME_{RAM} & & \\
LD & 11100\ |\ IMM_{Flag} [14]\ |\ R_{Dest} [13:11]\ |\ DC [10:8]\ |\ R_{Src2} [2:0]\ or\ \#IMM [7:0] & \text{Load value from memory to register} \\[4pt]
ST & 11101\ |\ IMM_{Flag} [14]\ |\ R_{Dest} [13:11]\ |\ DC [10:8]\ |\ R_{Src2} [2:0]\ or\ \#IMM [7:0] & \text{Store register to memory} \\[4pt]
\hline
\text{Do in } WB & & \\
MOV & 11110\ |\ IMM_{Flag} [14]\ |\ R_{Dest} [13:11]\ |\ R_src1 [2:0]\ or\ #IMM [7:0] & \text{Move value from} R_{Src1} \text{ or } \#IMM \text{ into } R_{Src1} \\[4pt]
\hline
\end{array}
\]
}
\end{center}

\newpage

\subsection{Control Unit ROM Data}

This is what we loaded into Control Unit (CU).

\noindent \texttt{CU\_ROM.hex}

\inputminted[fontsize=\scriptsize, breaklines, breakanywhere, breakindent=1em]{hex}{../test/CU_ROM.hex}

\newpage

\subsection{Assembler}

We had written an assembler in Python. \\

\noindent \texttt{compiler.py}

\inputminted[fontsize=\scriptsize, breaklines, breakanywhere, breakindent=1em]{py}{../test/compiler.py}

\newpage

\chapter{Installation and Execution Guide}

\section{Prerequisites}
\begin{itemize}
    \item Have git install in your system
    \item Have Hneemann's Digital installed in your system
\end{itemize}

\section{Git Clone from the Remote Repository}
\begin{lstlisting}[language=Bash ,basicstyle=\footnotesize\ttfamily]
git clone https://github.com/Pottarr/8Bit-CPU
\end{lstlisting}

After that you can open the \texttt{CPU.dig} in through your Digital.
\section{Test Assembly Code}

\subsection{Test 1: Overall Instructions}
\noindent \texttt{test1.ass}
\inputminted[fontsize=\scriptsize, breaklines, breakanywhere, breakindent=1em]{}{../test/test1.ass}

\newpage

\subsection{Test 2: Data Hazard Check}
\noindent \texttt{test2.ass}
\inputminted[fontsize=\scriptsize, breaklines, breakanywhere, breakindent=1em]{}{../test/test2.ass}

\subsection{Test 3: Structure Hazard and Control Hazard Check}
\noindent \texttt{test3.ass}
\inputminted[fontsize=\scriptsize, breaklines, breakanywhere, breakindent=1em]{}{../test/test3.ass}

\subsection{Test 4: Stack Warning Check}
\noindent \texttt{test4.ass}
\inputminted[fontsize=\scriptsize, breaklines, breakanywhere, breakindent=1em]{}{../test/test4.ass}

\newpage

\subsection{Test 5: I/O Test}
\noindent \texttt{test5.ass}
\inputminted[fontsize=\scriptsize, breaklines, breakanywhere, breakindent=1em]{}{../test/test5.ass}

\newpage

\chapter{Summary}

\section{Project Summary}

The \textit{Design and Implementation of a Minimal 8-Bit CPU} project focuses on creating a functional pipelined processor using the \textit{Digital} simulation software. 
This project integrates theoretical and practical knowledge of computer architecture by designing a five-stage pipelined CPU that executes a custom instruction set architecture (ISA). 
It emphasizes modular design, digital logic integration, and simulation-based verification of CPU functionalities such as arithmetic operations, branching, and interrupt handling.

\section{Learning Outcomes}

\begin{itemize}
    \item Successfully implemented a fully functional 8-bit pipelined CPU with distinct stages: Fetch, Decode, Execute, Memory, and Write Back.
    \item Developed a custom instruction set architecture (ISA) supporting arithmetic, logic, branching, and I/O operations.
    \item Designed and simulated essential CPU modules including the ALU, control unit, registers, and memory units (ROM and RAM).
    \item Implemented flag registers (Zero, Non-Zero, Carry) to support conditional branching and status tracking.
    \item Demonstrated interrupt handling and basic I/O communication using input and output ports.
    \item Verified CPU functionality using test programs, simulation waveforms, and step-by-step execution tracing.
\end{itemize}

\section*{Accomplishments}

\begin{itemize}
    \item Gained hands-on experience in digital logic and CPU design principles through simulation.
    \item Applied theoretical knowledge of pipelining and instruction execution to a practical, working model.
    \item Improved understanding of hardware design challenges such as data hazards and control flow management.
    \item Strengthened teamwork, modular circuit design, and documentation skills.
    \item Completed a comprehensive report and video demonstration showcasing CPU functionality and performance.
\end{itemize}

\newpage

% References/Bibliography section
\chapter{References}
% Include your references here in the proper LaTeX format, or use a BibTeX file
\begin{itemize}
    \item  \href{https://github.com/Pottarr/KMITL-SE-Classes-Archive/blob/main/SE16/Year2/Semester1/Computer-Organization-and-Architecture/Digital%20Design%20and%20Computer%20Architecture%20ARM%20Edition.pdf}{Digital Design and Computer Architecture ARM Edition}
    \item  \href{https://www.youtube.com/watch?v=6u2rzB3HtXM}{How to build a computer TUTORIAL
}

\end{itemize}

\newpage

\chapter{Appendix}

\section{Github Repositories}
\begin{itemize}
    \item This project repository: \url{https://github.com/Pottarr/8Bit-CPU}
    \item Helmut Neemann's Digital repository: \url{https://github.com/hneemann/Digital}
\end{itemize}
% \section{Test Cases Pipeline Stages Demonstration}
%
% \subsection{Test 0}
%
% \subsection{Test 1}
% \newpage
%
% \subsection{Test 2}
% \rotatebox{-90}{
%       \[
%     \begin{array}{c|cccccccccccccccccccc}
%         \hline
%         I_{C} & IF & ID & EX & ME & WB & \ & \ & \ & \ & \ & \ & \ & \ & \ & \ & \ & \ & \ & \ \\
%         I_{B} & \ & IF & ID & EX & ME & WB & \ & \ & \ & \ & \ & \ & \ & \ & \ & \ & \ & \ & \ \\
%         I_{A} & \ & \ & IF & ID & EX & ME & WB & \ & \ & \ & \ & \ & \ & \ & \ & \ & \ & \ & \ \\
%         I_{9} & \ & \ & IF & ID & EX & ME & WB & \ & \ & \ & \ & \ & \ & \ & \ & \ & \ & \ & \ \\
%         I_{7} & \ & \ & \ & IF & ID & EX & ME & WB & \ & \ & \ & \ & \ & \ & \ & \ & \ & \ & \ \\
%         I_{6} & \ & \ & \ & \ & IF & ID & EX & ME & WB & \ & \ & \ & \ & \ & \ & \ & \ & \ & \ \\
%         I_{5} & \ & \ & \ & \ & \ & IF & ID & EX & ME & WB & \ & \ & \ & \ & \ & \ & \ & \ & \ \\
%         I_{4} & \ & \ & \ & \ & \ & \ & IF & ID & EX & ME & WB & \ & \ & \ & \ & \ & \ & \ & \ \\
%         I_{3} & \ & \ & \ & \ & \ & \ & \ & IF & ID & EX & ME & WB & \ & \ & \ & \ & \ & \ & \ \\
%         I_{2} & \ & \ & \ & \ & \ & \ & \ & \ & IF & ID & EX & ME & WB & \ & \ & \ & \ & \ & \ \\
%         I_{1} & \ & \ & \ & \ & \ & \ & \ & \ & \ & IF & ID & EX & ME & WB & \ & \ & \ & \ & \ \\
%         I_{0} & \ & \ & \ & \ & \ & \ & \ & \ & \ & \ & IF & ID & EX & ME & WB & \ & \ & \ & \ \\
%         I_{0} & \ & \ & \ & \ & \ & \ & \ & \ & \ & \ & \ & IF & ID & EX & ME & WB & \ & \ & \ \\
%         I_{0} & \ & \ & \ & \ & \ & \ & \ & \ & \ & \ & \ & \ & IF & ID & EX & ME & WB & \ & \ \\
%         I_{0} & \ & \ & \ & \ & \ & \ & \ & \ & \ & \ & \ & \ & \ & IF & ID & EX & ME & WB & \ \\
%         I_{0} & \ & \ & \ & \ & \ & \ & \ & \ & \ & \ & \ & \ & \ & \ & IF & ID & EX & ME & WB \\
%     \end{array}
%     \]
% }
%
\end{document}
